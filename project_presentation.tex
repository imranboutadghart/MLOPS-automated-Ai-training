\documentclass[aspectratio=169]{beamer}
\usepackage[utf8]{inputenc}
\usepackage{graphicx}
\usepackage{booktabs}
\usepackage{xcolor}

% Theme
\usetheme{Madrid}
\usecolortheme{default}

\title{Automated MLOps Pipeline}
\subtitle{Continuous Model Training and Deployment}
\author{MLOps Engineering Project}
\date{December 30, 2025}
\institute{Machine Learning Operations}

\begin{document}

% Title Slide
\begin{frame}
\titlepage
\end{frame}

% ==================== SECTION 1: INTRODUCTION ====================
\section{Introduction}

\begin{frame}{Project Overview}
\begin{block}{What is MLOps?}
End-to-end automated ML pipeline bridging model development and production deployment
\end{block}

\vspace{0.5cm}

\begin{columns}
\column{0.5\textwidth}
\textbf{Key Features:}
\begin{itemize}
    \item Automated continuous training
    \item Distributed GPU training
    \item Experiment tracking
    \item Model versioning \& registry
\end{itemize}

\column{0.5\textwidth}
\textbf{Results:}
\begin{itemize}
    \item 80\% accuracy
    \item Sub-ms inference
    \item Production-ready API
    \item Full reproducibility
\end{itemize}
\end{columns}
\end{frame}

% ==================== SECTION 2: TECHNOLOGY STACK ====================
\section{Technology Stack}

\begin{frame}{Technology Stack}
\begin{table}
\centering
\small
\begin{tabular}{@{}ll@{}}
\toprule
\textbf{Component} & \textbf{Technology} \\ 
\midrule
Orchestration & Apache Airflow 2.8+ \\
Experiment Tracking & MLflow 2.8.1 \\
Containerization & Docker + Docker Compose \\
Model Framework & PyTorch 2.1+ \\
Distributed Training & HuggingFace Accelerate \\
Model Serving & FastAPI + Uvicorn \\
Object Storage & MinIO (S3-compatible) \\
Databases & PostgreSQL (2 instances) \\
Task Queue & Redis + Celery \\
Programming Language & Python 3.10 \\ 
\bottomrule
\end{tabular}
\end{table}
\end{frame}

% ==================== SECTION 3: ARCHITECTURE ====================
\section{System Architecture}

\begin{frame}{Architecture \& Pipeline}
\begin{columns}
\column{0.5\textwidth}
\textbf{Infrastructure:}
\begin{itemize}
    \item Airflow Cluster (orchestration)
    \item MLflow Server (tracking)
    \item MinIO (S3 storage)
    \item PostgreSQL (2 instances)
    \item FastAPI (model serving)
\end{itemize}

\vspace{0.3cm}
\footnotesize{9 Docker containers, 11 tools}

\column{0.5\textwidth}
\textbf{6-Stage Pipeline:}
\begin{enumerate}
    \item Data Validation
    \item Preprocessing (Pandas)
    \item Training (PyTorch + Accelerate)
    \item Evaluation (metrics)
    \item Registration (MLflow)
    \item Deployment (REST API)
\end{enumerate}
\end{columns}
\end{frame}

% ==================== SECTION 4: IMPLEMENTATION ====================
\section{Implementation}

\begin{frame}{Key Implementation Details}
\begin{table}
\centering
\small
\begin{tabular}{@{}lll@{}}
\toprule
\textbf{Service} & \textbf{Purpose} & \textbf{Port} \\ 
\midrule
airflow-webserver & Web UI & 8081 \\
airflow-scheduler & DAG scheduler & - \\
airflow-worker & Task executor & - \\
airflow-triggerer & Deferred tasks & - \\
mlflow & Tracking server & 5000 \\
minio & Artifact storage & 9000-9001 \\
model-serving & REST API & 8000 \\
postgres (2x) & Metadata DBs & - \\
redis & Task queue & - \\ 
\bottomrule
\end{tabular}
\end{table}
\end{frame}

% ==================== SECTION 5: RESULTS ====================
\section{Results}

\begin{frame}{Results: Diabetes Classification}
\begin{columns}
\column{0.5\textwidth}
\begin{table}
\centering
\begin{tabular}{@{}lc@{}}
\toprule
\textbf{Metric} & \textbf{Value} \\ 
\midrule
\textbf{Accuracy} & \textbf{80.0\%} \\
F1 Score & 75.2\% \\
Precision & 76.8\% \\
Recall & 73.7\% \\
\bottomrule
\end{tabular}
\end{table}

\column{0.5\textwidth}
\begin{table}
\centering
\begin{tabular}{@{}lc@{}}
\toprule
\textbf{Metric} & \textbf{Value} \\ 
\midrule
ROC-AUC & 0.84 \\
Training Time & ~45s \\
Inference Latency & 0.7-30ms \\
\bottomrule
\end{tabular}
\end{table}
\end{columns}

\vspace{0.5cm}
\begin{alertblock}{Production Ready}
Sub-millisecond latency with 80\% accuracy on medical diagnosis task
\end{alertblock}
\end{frame}

\begin{frame}[fragile]{API Performance}
\textbf{REST API Endpoints:}

\begin{lstlisting}[language=bash,basicstyle=\ttfamily\tiny]
# Health Check
GET http://localhost:8000/health
{
    "status": "healthy",
    "model_loaded": true,
    "model_version": "3",
    "timestamp": "2025-12-30T22:18:57"
}

# Prediction Request
POST http://localhost:8000/predict
Body: {"features": [6, 148, 72, 35, 0, 33.6, 0.627, 50]}
{
    "prediction": 1,
    "probabilities": [0.0, 1.0],
    "model_version": "3",
    "latency_ms": 0.74
}
\end{lstlisting}
\end{frame}

% ==================== SECTION 7: MLOPS PRACTICES ====================
\section{MLOps Best Practices}

\begin{frame}{Best Practices Implemented}
\begin{columns}
\column{0.5\textwidth}
\textbf{Version Control:}
\begin{itemize}
    \item Code in Git
    \item Models in MLflow Registry
    \item Docker image tagging
    \item Configuration as code
\end{itemize}

\vspace{0.3cm}
\textbf{Reproducibility:}
\begin{itemize}
    \item Fixed random seeds (42)
    \item Docker environments
    \item Pinned dependencies
    \item Logged hyperparameters
\end{itemize}

\column{0.5\textwidth}
\textbf{Monitoring:}
\begin{itemize}
    \item Structured logging
    \item MLflow metrics tracking
    \item Airflow UI monitoring
    \item API health endpoints
\end{itemize}

\vspace{0.3cm}
\textbf{CI/CD:}
\begin{itemize}
    \item Automated testing
    \item Docker builds
    \item Config-driven deployment
    \item Multiple strategies
\end{itemize}
\end{columns}
\end{frame}

% ==================== SECTION 8: DEPLOYMENT ====================
\section{Deployment}

\begin{frame}{Deployment Options}
\begin{block}{Current: Local Docker}
Production-ready serving via Docker Compose on localhost
\end{block}

\vspace{0.5cm}

\begin{columns}
\column{0.33\textwidth}
\textbf{AWS}
\begin{itemize}
    \item ECS/Fargate
    \item SageMaker
    \itcolumns}
\column{0.5\textwidth}
\textbf{Data Processing:}
\begin{itemize}
    \item Schema validation (30\% threshold)
    \item Median imputation
    \item StandardScaler normalization
    \item 70/15/15 train/val/test split
\end{itemize}

\column{0.5\textwidth}
\textbf{Training:}
\begin{itemize}
    \item MLP: 8-512-256-128-2
    \item AdamW optimizer, FP16
    \item Batch size: 64, Epochs: 10
    \item Training time: ~45 seconds
\end{itemize}
\end{columns}

\vspace{0.5cm}
\textbf{Deployment:} Canary (1\%→5\%→25\%→50\%→100\%) \& Shadow strategieservice mesh
\end{itemize}
\end{columns}
\end{frame}

% ==================== SECTION 9: FUTURE ====================
\section{Future Enhancements}

\begin{frame}{Future Enhancements}
\begin{columns}
\column{0.5\textwidth}
\textbf{Short-term:}
\begin{itemize}
    \item A/B testing framework
    \item Model explainability (SHAP/LIME)
    \item Data drift detection
    \item Enhanced monitoring dashboards
\end{itemize}

\column{0.5\textwidth}
\textbf{Long-term:}
\begin{itemize}
    \item Multi-model serving
    \item Feature store integration
    \item Advanced hyperparameter tuning
    \item Real-time streaming inference
    \item Kubernetes deployment
\end{itemize}
\end{columns}
\end{frame}

% ==================== SECTION 10: CONCLUSION ====================
\section{Conclusion}

\begin{frame}{Key Achievements}
\begin{block}{Production-Grade MLOps Pipeline}
Successfully implemented automated machine learning workflows with industry best practices
\end{block}

\vspace{0.3cm}

\textbf{Accomplishments:}
\begin{itemize}
    \item ✓ Fully automated continuous training pipeline
    \item ✓ 80\% accuracy on medical diagnosis task
    \item ✓ Sub-millisecond inference latency
    \item ✓ Docker-based deployment ready for cloud
    \item ✓ Complete experiment tracking and reproducibility
    \item ✓ Multiple deployment strategies implemented
\end{itemize}
\end{frame}

\begin{frame}{Summary}
\begin{columns}
\column{0.5\textwidth}
\textbf{System Capabilities:}
\begin{itemize}
    \item 6 pipeline stages
    \item 9 containerized services
    \item 11 integrated technologies
    \item Full automation
\end{itemize}

\column{0.5\textwidth}
\textbf{Production Features:}
\begin{itemize}
    \item Scalable architecture
    \item Cloud-ready deployment
    \item Monitoring \& logging
    \item Version control
\end{itemize}
\end{columns}

\vspace{0.5cm}

\begin{alertblock}{Foundation for Production ML Systems}
The platform demonstrates scalability, maintainability, and extensibility for enterprise machine learning operations.
\end{alertblock}
\end{frame}

% Final slide
\begin{frame}[plain]
\centering
\Huge Thank You!

\vspace{1cm}

\Large Questions?

\vspace{1cm}

\normalsize
\textbf{MLOps Engineering Project}\\
Automated Continuous Training and Deployment\\
December 30, 2025
\end{frame}

\end{document}
